\documentclass[]{elsarticle} %review=doublespace preprint=single 5p=2 column
%%% Begin My package additions %%%%%%%%%%%%%%%%%%%
\usepackage[hyphens]{url}
\usepackage{lineno} % add
\providecommand{\tightlist}{%
  \setlength{\itemsep}{0pt}\setlength{\parskip}{0pt}}

\bibliographystyle{elsarticle-harv}
\biboptions{sort&compress} % For natbib
\usepackage{graphicx}
\usepackage{booktabs} % book-quality tables
%% Redefines the elsarticle footer
%\makeatletter
%\def\ps@pprintTitle{%
% \let\@oddhead\@empty
% \let\@evenhead\@empty
% \def\@oddfoot{\it \hfill\today}%
% \let\@evenfoot\@oddfoot}
%\makeatother

% A modified page layout
\textwidth 6.75in
\oddsidemargin -0.15in
\evensidemargin -0.15in
\textheight 9in
\topmargin -0.5in
%%%%%%%%%%%%%%%% end my additions to header

\usepackage[T1]{fontenc}
\usepackage{lmodern}
\usepackage{amssymb,amsmath}
\usepackage{ifxetex,ifluatex}
\usepackage{fixltx2e} % provides \textsubscript
% use upquote if available, for straight quotes in verbatim environments
\IfFileExists{upquote.sty}{\usepackage{upquote}}{}
\ifnum 0\ifxetex 1\fi\ifluatex 1\fi=0 % if pdftex
  \usepackage[utf8]{inputenc}
\else % if luatex or xelatex
  \usepackage{fontspec}
  \ifxetex
    \usepackage{xltxtra,xunicode}
  \fi
  \defaultfontfeatures{Mapping=tex-text,Scale=MatchLowercase}
  \newcommand{\euro}{€}
\fi
% use microtype if available
\IfFileExists{microtype.sty}{\usepackage{microtype}}{}
\ifxetex
  \usepackage[setpagesize=false, % page size defined by xetex
              unicode=false, % unicode breaks when used with xetex
              xetex]{hyperref}
\else
  \usepackage[unicode=true]{hyperref}
\fi
\hypersetup{breaklinks=true,
            bookmarks=true,
            pdfauthor={},
            pdftitle={Segmentation of Gadolinium-Enhancing Lesions},
            colorlinks=true,
            urlcolor=blue,
            linkcolor=magenta,
            pdfborder={0 0 0}}
\urlstyle{same}  % don't use monospace font for urls
\setlength{\parindent}{0pt}
\setlength{\parskip}{6pt plus 2pt minus 1pt}
\setlength{\emergencystretch}{3em}  % prevent overfull lines
\setcounter{secnumdepth}{0}
% Pandoc toggle for numbering sections (defaults to be off)
\setcounter{secnumdepth}{0}
% Pandoc header


\usepackage[nomarkers]{endfloat}

\begin{document}
\begin{frontmatter}

  \title{Segmentation of Gadolinium-Enhancing Lesions}
    \author[JHU]{John Muschelli\corref{c1}}
   \ead{jmusche1@jhu.edu} 
   \cortext[c1]{Corresponding Author}
    \author[JHU]{Adrian Gherman}
   \ead{adig@jhu.edu} 
  
    \author[JHU]{Brian S. Caffo}
   \ead{bcaffo@jhsph.edu} 
  
    \author[JHU]{Ciprian M. Crainiceanu}
   \ead{ccraini1@jhu.edu} 
  
      \address[JHU]{Johns Hopkins Bloomberg School of Public Health, Department of
Biostatistics, 615 N Wolfe St, Baltimore, MD, 21205}
  
  \begin{abstract}
  This is the abstract.
  
  It consists of two paragraphs.
  \end{abstract}
  
 \end{frontmatter}

\section{Introduction}\label{introduction}

Neuroimaging research has been increasing in popularity. There are X
number of studies in fMRI and XX in other structural imaging. The
NeuroImage journal, one of the most prestigious (not right word)
journals for neuroimaging is dominated by analyses that were in bash
using third party software, NiPype, or MATLAB. Python reproducibility
has increased with the introduction of Jupyter IPython notebooks. Muhc
of this development has mirrored the tools in R, which have some of the
most state-of-the-art reproducibility tools. Also, similar to PYthon, R
is free and open source.

\begin{itemize}
\tightlist
\item
  Neuroimaging work more common
\item
  Reproducibility problems

  \begin{itemize}
  \tightlist
  \item
    R has some of the best reproducible tools
  \end{itemize}
\end{itemize}

R is increasing in popularity, with there being over 9,000 package now
on the Comprehensive R Archive Network (CRAN). R has a strong package
system, which has a large system of checks to ensure a large amount of
operability. Although CRAN has a tested and stable system, that has been
developed over the last 1000 years(?), there are some aspects of the
CRAN checking system that does not work for neuroimaging packages.

\subsection{Additional Checks}\label{additional-checks}

\subsubsection{Third Party Software}\label{third-party-software}

\begin{itemize}
\tightlist
\item
  FSL
\item
  AFNI
\item
  FREESURFER
\end{itemize}

\subsection{TRAVIS}\label{travis}

\section{Bioconductor}\label{bioconductor}

In 2004, the Bioconductor system enabled the bioinformatics and genomics
work of R users to be more integrated and systemized (Gentleman et al.
2004).

\subsection{Data Packages}\label{data-packages}

Like Bioconductor, we need data packages that allow users to test
software and examples on.

\section*{References}\label{references}
\addcontentsline{toc}{section}{References}

\hypertarget{refs}{}
\hypertarget{ref-gentleman2004bioconductor}{}
Gentleman, Robert C, Vincent J Carey, Douglas M Bates, Ben Bolstad,
Marcel Dettling, Sandrine Dudoit, Byron Ellis, et al. 2004.
``Bioconductor: Open Software Development for Computational Biology and
Bioinformatics.'' \emph{Genome Biology} 5 (10). BioMed Central: 1.

\end{document}


