\documentclass[]{elsarticle} %review=doublespace preprint=single 5p=2 column
%%% Begin My package additions %%%%%%%%%%%%%%%%%%%
\usepackage[hyphens]{url}
\usepackage{lineno} % add
\providecommand{\tightlist}{%
  \setlength{\itemsep}{0pt}\setlength{\parskip}{0pt}}

\bibliographystyle{elsarticle-harv}
\biboptions{sort&compress} % For natbib
\usepackage{graphicx}
\usepackage{booktabs} % book-quality tables
%% Redefines the elsarticle footer
%\makeatletter
%\def\ps@pprintTitle{%
% \let\@oddhead\@empty
% \let\@evenhead\@empty
% \def\@oddfoot{\it \hfill\today}%
% \let\@evenfoot\@oddfoot}
%\makeatother

% A modified page layout
\textwidth 6.75in
\oddsidemargin -0.15in
\evensidemargin -0.15in
\textheight 9in
\topmargin -0.5in
%%%%%%%%%%%%%%%% end my additions to header

\usepackage[T1]{fontenc}
\usepackage{lmodern}
\usepackage{amssymb,amsmath}
\usepackage{ifxetex,ifluatex}
\usepackage{fixltx2e} % provides \textsubscript
% use upquote if available, for straight quotes in verbatim environments
\IfFileExists{upquote.sty}{\usepackage{upquote}}{}
\ifnum 0\ifxetex 1\fi\ifluatex 1\fi=0 % if pdftex
  \usepackage[utf8]{inputenc}
\else % if luatex or xelatex
  \usepackage{fontspec}
  \ifxetex
    \usepackage{xltxtra,xunicode}
  \fi
  \defaultfontfeatures{Mapping=tex-text,Scale=MatchLowercase}
  \newcommand{\euro}{€}
  \newcommand{\fixme}[1]{{\color{red} #1}}
\fi
% use microtype if available
\IfFileExists{microtype.sty}{\usepackage{microtype}}{}
\ifxetex
  \usepackage[setpagesize=false, % page size defined by xetex
              unicode=false, % unicode breaks when used with xetex
              xetex]{hyperref}
\else
  \usepackage[unicode=true]{hyperref}
\fi
\hypersetup{breaklinks=true,
            bookmarks=true,
            pdfauthor={},
            pdftitle={Segmentation of Gadolinium-Enhancing Lesions},
            colorlinks=true,
            urlcolor=blue,
            linkcolor=magenta,
            pdfborder={0 0 0}}
\urlstyle{same}  % don't use monospace font for urls
\setlength{\parindent}{0pt}
\setlength{\parskip}{6pt plus 2pt minus 1pt}
\setlength{\emergencystretch}{3em}  % prevent overfull lines
\setcounter{secnumdepth}{0}
% Pandoc toggle for numbering sections (defaults to be off)
\setcounter{secnumdepth}{0}
% Pandoc header
  \newcommand{\fixme}[1]{{\color{red} #1}}


\usepackage[nomarkers]{endfloat}

\begin{document}
\begin{frontmatter}

  \title{The Neuroconductor project}
    \author[JHU]{John Muschelli\corref{c1}}
   \ead{jmusche1@jhu.edu} 
   \cortext[c1]{Corresponding Author}
    \author[JHU]{Adrian Gherman}
   \ead{adig@jhu.edu} 
  
    \author[JHU]{Brian S. Caffo}
   \ead{bcaffo@jhsph.edu} 
  
    \author[JHU]{Ciprian M. Crainiceanu}
   \ead{ccraini1@jhu.edu} 
  
      \address[JHU]{Johns Hopkins Bloomberg School of Public Health, Department of
Biostatistics, 615 N Wolfe St, Baltimore, MD, 21205}
  
  \begin{abstract}
  The Neuroconductor project is an initiative for the collaborative
  creation of extensible software for computational imaging analysis, with
  a focus on neuroimaging. The goals of the project include: integrate
  fast and collaborative development of tested software, increasing
  reproducibility in analyses of imaging data, and promoting the
  achievement of remote reproducibility of research results. We describe
  details of our goals, identify the current implementation and current
  challenges, and provide examples of existing software and how they would
  integrate into the Neuroconductor system. \fixme{The Neuroconductor project offers an easy and unifited framework in R for the preprocessing and advanced statistical analysis of imaging data.  }
  \end{abstract}
  
 \end{frontmatter}

% Introduction
\section{Introduction}\label{introduction}

Neuroimaging research has been increasing in popularity. There are \fixme{XX}
number of studies in fMRI and \fixme{XX} in other structural imaging. \fixme{Talk about big projects like ADNI, ABIDE, 1000 Functional connectomes, etc}.  The
NeuroImage journal, one of the most prestigious (not right word)
journals for neuroimaging is dominated by analyses that were in bash
using third party software, NiPype, or MATLAB. Python reproducibility
has increased with the introduction of Jupyter IPython notebooks. Much
of this development has mirrored the tools in R, which have some of the
most state-of-the-art reproducibility tools. Also, similar to PYthon, R
is free and open source.

\begin{itemize}
\tightlist
\item
  Neuroimaging work more common
\item
  Reproducibility problems

  \begin{itemize}
  \tightlist
  \item
    R has some of the best reproducible tools
  \end{itemize}
\end{itemize}

R is increasing in popularity, with there being over 9,000 package now
on the Comprehensive R Archive Network (CRAN). R has a strong package
system, which has a large system of checks to ensure a large amount of
operability. Although CRAN has a tested and stable system, that has been
developed over the last 1000 years(?), there are some aspects of the
CRAN checking system that does not work for neuroimaging packages.

\fixme{Several of the statistical challenges encountered in the analysis of imaging data are shared with the field of genomics. An example of a statistical problem common to both imaging and genomic data is the presence of between-subject variability that is technical in nature, for instance the well-known batch effects (CITE) seen in genomics, due to different preprocessing dates, similar to between-scanner variability seen in large multi-site imaging studies (CITE). I want to say something like the Neuroconductor project will allow to integrate well-developed and robust statistical methods for handling such challenges, and it will make easy to adapt genomic tools to the world of brain imaging.

Talk about ComBat, SVA, and limma; maybe mention how popular they are by number of citations? Wondering if we could have one showcase section where we show that just using the Empirical Bayes function in limma automatically improves the statistical analysis of brain imaging data; that would sell Neuroconductor big deal. }


\fixme{Other note: mention the cluster PNAS paper; we want to avoid such situations in the future; we need clear, reproducible and well-implemented statistical methods, and R is the perfect platform for that. We need to think: what can we do better in R that no other language/software can offer?}


\subsection{Additional Checks}\label{additional-checks}

\subsubsection{Third Party Software}\label{third-party-software}

\begin{itemize}
\tightlist
\item
  FSL
\item
  AFNI
\item
  FREESURFER
\item
  SPM
\end{itemize}

We will refer to R-Forge, OmegaHat, Bioconductor, and CRAN as standard
repositories.

\subsection{Devtools}\label{devtools}

In YEAR, Hadley and RStudio had published the \texttt{devtools} package.
The \texttt{devtools} package provided the tools to install R packages
from a multitude of sources. The Neuroconductor relies on the
installation script for R packages on GitHub. Moreover, it allowed for
the introduction of a flag in the installation of an R package (the
\texttt{Remotes:} field) that allowed users specify a dependency for the
package that can be located on a source that is not a standard
repository. Previous to this, if a package depended on a package that
was not in a standard repository, the user would have to manually
install that dependency before installing the package in question.

In addition to installing packages from GitHub, there are additional
options to the installer scripts, which allow users to install specific
snapshots of these packages. These snapshots can be based on GitHub
commit identifiers (IDs), tags, or referrences.

The \texttt{remotes} package provides a lightweight version of the
\texttt{devtools} package for installing from non-standard and standard
sources.

In addition to the install functions, the \texttt{devtools} package
allows for a up-to-date R package development system. The RStudio IDE
integrates the \texttt{devtools} package so that R package development
can be done in a more standardized way.

\subsection{GitHub}\label{github}

GitHub APi. Git vs.~svn.

\subsection{TRAVIS}\label{travis}

Seperately from any development in the R community, continuous
integration (CI) services have become largely available that allows for
automated building and checking of software. The Travis CI is a
``hosted, distributed continuous integration service used to build and
test software projects hosted at GitHub''. In conjuction with the R
community, Travis CI has configured the ability to seamlessly check R
packages on multiple systems. In addition to the GitHub API described
above, the Travis CI API provides an automated system for checking R
package installation.

\subsubsection{DRAT}\label{drat}

The \texttt{drat} (Drat R Archive Template) package has provided a
template to set up a repository similar to CRAN mirrors and other
standard sources. One of the large benefits of using a drat repository
is that users can use the default way of installing packages in R
(\texttt{install.packages}) versus that from \texttt{devtools} (e.g.
\texttt{install\_github}) and requires no additional dependencies such
as \texttt{devtools} or \texttt{remotes}.

One good example of a drat repository is
\href{https://ropensci.org/}{ROpenSci}, which has a series of packages
that are based on GitHub. The only additional step for installing from a
drat repository is to specify the \texttt{repos} argument in
\texttt{install.packages}:
\texttt{install.packages("package\_name",\ repos="http://path/to/drat/repo")}.

% Data packages
\section{Data packages and atlases}\label{datapackages}

\section{Bioconductor}\label{bioconductor}

In 2004, the Bioconductor system enabled the bioinformatics and genomics
work of R users to be more integrated and systemized (Gentleman et al.
2004).

\subsection{Data Packages}\label{data-packages}

Like Bioconductor, we need data packages that allow users to test
software and examples on.

\section*{References}\label{references}
\addcontentsline{toc}{section}{References}

\hypertarget{refs}{}
\hypertarget{ref-gentleman2004bioconductor}{}
Gentleman, Robert C, Vincent J Carey, Douglas M Bates, Ben Bolstad,
Marcel Dettling, Sandrine Dudoit, Byron Ellis, et al. 2004.
``Bioconductor: Open Software Development for Computational Biology and
Bioinformatics.'' \emph{Genome Biology} 5 (10). BioMed Central: 1.

% Table 
\begin{table}[!ht]
\footnotesize
\centering
\caption{\textbf{Initial Neuroconductor packages.}}\label{tab:summary}
\begin{tabular}{lllcl}
\hline \\[-2ex]
\textbf{Package} & \textbf{Description} & \textbf{Popular functions and applications} & \textbf{Reference} \\
\hline \\ [-1.5ex]
\texttt{oro.nifti} &  &  & &   \\ [1ex]
\texttt{freesurfer} & &  & \\ [1ex]
 \texttt{fslr} &  & &   \\ [1ex]
 \texttt{brainR} &  &  &    \\ [1ex]
 \texttt{papayar} &  &  &    \\ [1ex]
   \texttt{ANTsR} & R Package for ANTS &   \\ [1ex]
  \texttt{drammsr} & R Package for DRAMMS & 
  DRAMMS registration, RAVENS maps  \\ [1ex]
 \texttt{WhiteStripe} & Intensity normalization with White Stripe & & &   \\ [1ex]
   \texttt{RAVEL} & Stratified quantile normalization &&&  \\ [1ex]
     \texttt{oasisr} &  &&&  \\ [1ex]
     \texttt{shrinkR} &  &&&  \\ [1ex]
\hline
\end{tabular}
\end{table}

\end{document}


